%%
%% Author: User1
%% 26.04.2018
%%

% Preamble
\documentclass[a4paper,10pt]{article}

% Packages
\usepackage{mathtext}
\usepackage[T2A]{fontenc}
\usepackage[utf8]{inputenc}
\usepackage[russian]{babel}
\usepackage{amsmath}
\usepackage{amsfonts}
\usepackage{longtable}
\usepackage{amssymb}
\usepackage{graphicx}
\usepackage[left=2cm,right=2cm,
    top=2cm,bottom=2cm,bindingoffset=0cm]{geometry}
\usepackage{color}
\usepackage{gensymb}

\usepackage{enumitem}
\setlist[enumerate]{label*=\arabic*.}

\usepackage{indentfirst}

\usepackage{titlesec}

% Document
\begin{document}

    \section{Поступенчатый расчет компрессора.}

    

    \subsection{Первая ступень}
    \subsubsection{Исходные данные.}

    
    \begin{enumerate}

        \item Коэффициент напора ступени: $\bar{H}_{т} = 0.32$.
        \item Коэффициент напора следующей ступени: $ \bar{H}_{т\ след} = 0.334$.
        \item Окружная скорость на конце рабоче лопатки: $ u_{1к} = 363.61\ м/с $.
        \item Поправочный коэффициент для учета влияния вязкости робочего тела у втулки и корпуса: $ k_H = 0.98 $.
        \item Адиабатический КПД ступени: $ \eta_{ад}^* = 0.85 $.
        \item Относительный диаметр втулки на входе: $ \bar{d}_{1вт} = 0.5 $.
        \item Коэффициент расхода на входе в ступень: $ \bar{c}_{1a} = 0.45 $.
        \item Коэффициент расхода на выходе из ступени: $ \bar{c}_{3a} = 0.462 $.
        \item Степень реактивности на среднем радиусе ступени: $ R_{ср} = 0.5 $.
        \item Степень реактивности на среднем радиусе следующей ступени: $ R_{ср\ след} = 0.5 $.
        \item Температура торможения на входе в ступень: $ T_1^* = 288\ К $.
        \item Давление торможения на входе в ступень: $ p_1^* = 0.0993\ МПа $.
        \item Расход на входе в ступень: $ G = 40\ кг/с $.
        \item Частота вращения на входе в ступень: $ n = 11000\ об/мин $.
        \item Параметр, определяющий положения постоянного диаметра ступени: $ p_{пост} = 0.5 $.


    \end{enumerate}
    

    \subsubsection{Расчет.}

    

    

    \begin{enumerate}

        \item Средняя теплоемкость в процессе сжатия в компрессоре:

        \[
        	c_{p\ ср} = \frac{k_{ср} \cdot R}{k_{ср} - 1} = 
        	\frac{1.386 \cdot 287.4 }{ 1.386 - 1 } =
			1032.16\ Дж/кг.
        \]
		
		\item Осевая скорость на входе в рабочее колесо:
		
		\[
			c_{1a} = u_{1к} \cdot \bar{c}_{1a} = 
			363.61 \cdot 0.45 = 163.62\ м/с.
		\]

        \item Теоретический напор ступени:

        \[
            H_т = \bar{H}_т \cdot u_{1к}^2 = 
            0.32 \cdot 363.61^2 = 
            42.3068\ КДж/кг.
        \]

        \item Действительная работа сжатия:
        \[
            L_z = k_H \cdot H_т = 0.98 \cdot 42.3068 = 
            41.4607\ КДж/кг.
        \]

        \item Адиабатическая работа сжатия:
        \[
            H_{ад} = L_z \cdot \eta_{ад}^* = 
            41.4607 \cdot 0.85 =
            35.2416\ КДж/кг.
        \]

        \item Повышение полной температуры в сутпени:
        \[
            \Delta T^* = \frac{L_z}{c_{p\ ср}} = 
            \frac{ 41.4607 \cdot 10^3 }{ 1032.16 } = 
            40.17\ К.
        \]

        \item Полная температура на выходе из ступени:
        \[
            T_3^* = T_1^* + \Delta T^* = 
            288 + 40.17 = 
            328.17\ К.
        \]

        \item Степень повышения полного давления в ступени:
        \[
            \pi^* = \left( 1 + \frac{ H_{ад} }{ c_{p\ ср} \cdot T_1^* } \right) ^ { \frac{ k_{ср} }{ k_{ср} - 1 } } = 
            \left( 
                1 + \frac{ 35.2416 \cdot 10^3 
                        }{ 1032.16 \cdot 288 } 
            \right) ^ 
            { \frac{ 1.386 }{ 1.386 - 1 } } =
            = 1.495
        \]

        \item Полное давление на выходе из ступени:
        \[
            p_3^* = p_1^* \cdot \pi^* = 
            0.0993 \cdot  = 
            0.1485\ МПа.
        \]

        \item Критическая скорость звука на входе в ступень:
        \[
            a_{кр1} = \sqrt{ \frac{2 \cdot k_{ср} }{ k_{ср} + 1 } \cdot R \cdot T_1^* } = 
            \sqrt{ 
                \frac{ 
                        2 \cdot 1.386 
                    }{ 
                        1.386 + 1 
                } \cdot 287.4 \cdot 288
                } = 
            310.09\ м/с.
        \]

        \item Критическая скорость звука на выходе из ступени:
        \[
            a_{кр3} = \sqrt{ \frac{2 \cdot k_{ср} }{ k_{ср} + 1 } \cdot R \cdot T_3^* } = 
            \sqrt{ 
                \frac{ 
                        2 \cdot 1.386 
                    }{ 
                        1.386 + 1 
                } \cdot 287.4 \cdot 328.17
                } = 
            331.01\ м/с.
        \] 

        \item Относительный средний радиус на входе в ступен:
        \[
            \bar{r}_{1ср} = \sqrt{ \frac{ 1 + \bar{d}_{1вт}^2 }{ 2 } } = 
            \sqrt{ \frac{ 1 + 0.5 ^ 2 }{ 2 } } = 
            0.791.
        \]

        \item Безразмерная окружная составляющая абсолютной скорости на входе:
        \[
            \bar{c}_{1u} = \bar{r}_{1ср} \cdot (1 - R_{ср}) - \frac{ \bar{H}_т }{ 2 \cdot  \bar{r}_{1ср}} = 
            0.791 \cdot (1 - 0.5) -
            \frac{ 0.32 }{ 2 \cdot  0.791} =
            0.193. 
        \]

        \item Окружная составляющая абсолютной скорости на входе:
        \[
            c_{1u} = \bar{c}_{1u} \cdot u_{1к} = 
            0.193 \cdot 363.61 =
            70.139\ м/с.
        \]

        \item Абсолютная скорость на входе:
        \[
            c_1 = \sqrt{ c_{1a}^2 + c_{1u}^2 } = 
            \sqrt{ 163.62^2 + 70.14^2 } =
            178.02\ м/с.
        \]

        \item Направление абсолютной скорости на входе:
        \[
            \alpha_1 = \arctan{ \frac{ \bar{c}_{1a} }{ \bar{c}_{1u} } } =
             \arctan{ \frac{ 0.45 }{ 0.193 } } = 
             66.797 ^\circ.
        \]

        \item Приведенная скорость на входе:
        \[
            \lambda_1 = \frac{ c_{1a} }{ \sin{\alpha_1} \cdot a_{кр1} } = 
            \frac{ 163.62 }{ \sin{66.797^\circ} \cdot 310.09 } = 
            0.5741.
        \]

        \item ГДФ расхода на входе:
        \[
            q_1 = q(\lambda_1, k_{ср}) = q(0.5741, 1.386) = 
            0.787.
        \]

        \item Кольцевая площадь на входе в ступень:
        \begin{gather*}
            F_1 = \frac{ G \cdot \sqrt{R \cdot T_1^*} }{ k_{ср} \cdot p_1^* \cdot q_1 \cdot \sin{\alpha_1} } =\\
            =\frac{ 
                40 \cdot \sqrt{ 287.4 \cdot 288}
            }{ 
                1.386 \cdot 0.0993 \cdot 10^6 \cdot 0.787 
                \cdot \sin{66.797^\circ} 
            } =\\
            =0.2348\ м^2.\\
        \end{gather*}

        \item Периферийный диаметр на входе:
        \[
            D_{1к} = \sqrt{ \frac{ 4 \cdot F_1 }{ \pi \cdot (1 - \bar{d}_{1вт}^2) } } = 
            \sqrt{ \frac{ 
                    4 \cdot 0.2348 
                }{ 
                    \pi \cdot (1 - 0.5^2) 
            } } = 
            0.6313\ м.
        \]

        \item Втулочный диаметр на входе:
        \[
            D_{1вт} = \bar{d}_{1вт} \cdot D_{1к} = 
            0.5 \cdot 0.6313 = 
            0.3157\ м.
        \]

        \item Постоянный диаметр:
        \begin{gather*}
            D_{пост} = D_{1к} \cdot 
                \sqrt{ \frac{ 
                        1 + \bar{d}_{1вт}^2 \cdot \frac{ 1 - p_{пост} }{ p_{пост} }  
                    }{
                        1 + \frac{ 1 - p_{пост} }{ p_{пост}}
                } } =\\
            =0.6313 \cdot 
                \sqrt{ \frac{ 
                        1 + 0.5^2 \cdot 
                        \frac{ 1 - 0.5 }{ 0.5 }  
                    }{
                        1 + \frac{ 1 - 0.5 }{ 0.5}
                } } =\\
                =0.4991\ м.\\  
        \end{gather*}

        \item Параметры на выходе находятся методом последовательных приближений. Представим результат последней итерации.
        \begin{enumerate}

            \item Угол на выходе с предпоследней итерации:
            \[
                \alpha_3^\prime = 65.287^\circ.
            \]

            \item Периферийная окружная скорость на выходе с предпоследней итерации:
            \[
                u_{3к}^\prime = 346.989\ м/с.
            \] 

            \item Осевая скорость на выходе из РК:
            \[
                c_{3a} = u_{3к}^\prime \cdot \bar{c}_{3a} = 
                346.989 \cdot 0.462 = 
                160.47\ м/с.
            \]

            \item Приведенная скорость на выходе из РК:
            \[
                \lambda_3 = \frac{ c_{3a} }{ \sin{\alpha_3^\prime} \cdot a_{кр3} } = 
                \frac{ 160.47 
                }{ 
                    \sin{65.287^\circ} \cdot 331.01 } =
                0.534. 
            \]

            \item ГДФ расхода на выходе:
            \[
                q_3 = q(\lambda_3, k_{ср}) = = q(0.534, 1.386) = 
                0.746.
            \]

            \item Кольцевая площадь на выходе:
            \begin{gather*}
                F_3 = \frac{ G \cdot \sqrt{R \cdot T_3^*} }{ k_{ср} \cdot p_3^* \cdot q_3 \cdot \sin{\alpha_3^\prime} } =\\ 
                =\frac{ 
                    40 \cdot \sqrt{ 287.4 \cdot 328.17}
                }{ 
                    1.386 \cdot 0.1485 \cdot 10^6 \cdot 0.746 
                    \cdot \sin{65.286^\circ} 
                } =\\ 
                =0.1788\ м^2.\\
            \end{gather*}

            \item Относительный диаметр втулки на выходе:
            \begin{gather*}
                \bar{d}_{3вт} = \sqrt{ \frac{ 
                                0.25 \cdot \pi \cdot \left( 
                                    1 + \frac{ 1 - p_{пост} }{ p_{пост} } 
                                \right) \cdot D_{пост}^2 - F_3
                        }{ 
                            F_3 \cdot \frac{ 1 - p_{пост} }{ p_{пост} } + 0.25 \cdot \pi \cdot \left( 
                                    1 + \frac{ 1 - p_{пост} }{ p_{пост} } 
                                \right) \cdot D_{пост}^2
                    } } =\\ 
                =\sqrt{ \frac{ 
                                0.25 \cdot \pi \cdot \left( 
                                    1 + \frac{ 1 - 0.5 }{ 0.5 } 
                                \right) \cdot 0.499^2 - 0.179
                        }{ 
                            0.179 \cdot 
                            \frac{ 1 - 0.5 }{ 0.5 } 
                            + 0.25 \cdot \pi \cdot \left( 
                                    1 + \frac{ 1 - 0.5 }{ 0.5 } 
                            \right) \cdot 0.499^2
                    } } =\\ 
                = 0.61.\\
            \end{gather*}

            \item Относительный средний диаметр на выходе:
            \[
                \bar{r}_{3ср} = \sqrt{ \frac{ 1 + \bar{d}_{3вт}^2 }{ 2 } } = 
                \sqrt{ \frac{ 1 + 0.61 ^ 2 }{ 2 } } = 
                0.828.
            \]

            \item Относительная окружная составляющая скорости на выходе из ступени:
            \[
                \bar{c}_{3u} = \bar{r}_{3ср} \cdot (1 - R_{ср\ след}) - \frac{ \bar{H}_{т\ след} }{ 2 \cdot  \bar{r}_{3ср}} = 
                0.828 \cdot (1 - 0.5) -
                \frac{ 0.334 }{ 2 \cdot  0.828} =
                0.213. 
            \]

            \item Новое значение угла потока на выходе:
            \[
                \alpha_3 = \arctan{ \frac{ \bar{c}_{3a} }{ \bar{c}_{3u} } } = 
                \arctan{ \frac{ 0.462 }{ 0.213 } } = 
                65.286^\circ.
            \]

            \item Периферийный диаметр на выходе:
            \[
                D_{3к} = \sqrt{ \frac{ 4 \cdot F_3 }{ \pi \cdot (1 - \bar{d}_{3вт}^2) } } = 
                \sqrt{ \frac{ 
                        4 \cdot 0.1788 
                    }{ 
                        \pi \cdot (1 - 0.61^2) 
                } } = 
                0.6024\ м.
            \]

            \item Втулочный диаметр на выходе:
            \[
                D_{3вт} = \bar{d}_{3вт} \cdot D_{3к} = 
                0.61 \cdot 0.6024 = 
                0.3677\ м.
            \]

            \item Новое значение окружной скорости на периферии на выходе:
            \[
                u_{3к} = \frac{\pi \cdot D_{3к} \cdot n }{ 60 } = 
                \frac{\pi \cdot 0.6024 \cdot 11000 }{ 60 } = 
                346.98\ м/с.
            \]

            \item Невязка по углу:
            \[
                \delta_{\alpha} = \frac{ \left| \alpha_3^\prime - \alpha_3 \right| }{ \alpha_3^\prime } \cdot 100 \% = 
                \frac{ 
                    \left| 65.29^\circ - 65.29^\circ \right| 
                }{ 
                    65.29^\circ
                } = 
                0.002 \%.
            \]

            \item Невязка по скорости:
            \[
                \delta_{u} = \frac{ \left| u_{3к}^\prime - u_{3к} \right| }{ u_{3к}^\prime } \cdot 100 \% = 
                \frac{ 
                    \left| 346.99 - 346.98 \right| 
                }{ 
                    19881.02                
                } = 
                0.002 \%.
            \]

        \end{enumerate}

        \item Окружная составляющая скорости на выходе:
        \[
            c_{3u} = \bar{c}_{3u} \cdot u_{3к} = 
            0.213 \cdot 346.98 = 
            73.85\ м/с.
        \]

        \item Абсолютная скорость на выходе из ступени:
        \[
            c_3 = \sqrt{ c_{3u}^2 + c_{3a}^2 } = 
            \sqrt{ 73.85^2 + 160.47^2 } =
            15.31\ м/с. 
        \] 

        \item Относительный средний диаметр на выходе из РК:
        \[
            \bar{r}_{2ср} = 0.5 \cdot ( \bar{r}_{1ср} + \bar{r}_{3ср} ) = 
            0.5 \cdot ( 0.791 + 0.828 ) = 
            0.81.
        \]

        \item Относительная окружная составляющая скорости на выходе из РК:
        \[
            \bar{c}_{2u} = \frac{ \bar{H}_т + \bar{c}_{1u} \cdot \bar{r}_{1ср} }{ \bar{r}_{2ср} } =
            \frac{ 
                0.32 + 0.193 \cdot 0.791
            }{ 
                0.81 
            } = 
            0.584. 
        \]

        \item Осевая составляющая скорости на выходе из РК:
        \[
            c_{2a} = 0.5 \cdot (c_{1a} + c_{3a} ) = 
            0.5 \cdot (163.622 + 160.465) = 
            162.044\ м/с.
        \]

        \item Периферийная окружная скорость на выходе из РК:
        \[
            u_{2к} = 0.5 \cdot (u_{1к} + u_{3к}) = 
            0.5 \cdot ( 363.61 + 346.98 ) = 
            355.294\ м/с.
        \]

        \item Относительная осевая скорость на выходе из РК:
        \[
            \bar{c}_{2a} = \frac{ c_{2a} }{ u_{2к} } = 
            \frac{ 162.04 }{ 355.29 } = 
            0.46.
        \]

        \item Угол потока в относительном движении на входе в РК:
        \[
            \beta_1 = \arctan{ \frac{ \bar{c}_{1a} }{ \bar{r}_{1ср} - \bar{c}_{1u} } } = 
            \arctan{ \frac{ 0.45 }{ 0.791 - 0.193 } } =
            36.977^\circ.
        \] 

        \item Угол потока в относительном движении на выходе из РК:
        \[
            \beta_2 = \arctan{ \frac{ \bar{c}_{2a} }{ \bar{r}_{2ср} - \bar{c}_{2u} } } = 
            \arctan{ \frac{ 0.456 }{ 0.81 - 0.584 } } =
            63.66^\circ.
        \] 

        \item Угол потока в абсолютном движении на выходе из РК:
        \[
            \alpha_2 = \arctan{ \frac{ \bar{c}_{2a} }{ \bar{c}_{2u} } } = 
            \arctan{ \frac{ 0.456 }{ 0.584 } } =
            38.0^\circ.
        \]

        \item Угол поворота потока в РК:
        \[
            \varepsilon_{рк} = \beta_2 - \beta_1 = 
            63.66^\circ - 36.98^\circ = 
            26.68^\circ.
        \]

         \item Угол поворота потока в НА:
        \[
            \varepsilon_{на} = \alpha_3 - \alpha_2 = 
            65.29^\circ - 38.0^\circ = 
            27.28^\circ.
        \]

        \item Относительная скорость на входе в РК:
        \[
            w_1 = \frac{ c_{1a} }{ \sin{\beta_1} } = 
            \frac{ 163.62 }{ \sin{ 36.98^\circ } } = 
            272.03\ м/с.
        \]

        \item Относительная скорость на выходе из РК:
        \[
            w_2 = \frac{ c_{2a} }{ \sin{\beta_2} } = 
            \frac{ 162.04 }{ \sin{ 63.66^\circ } } = 
            180.82\ м/с.
        \]

        \item Окружная составляющая относительной скорости на входе в РК:
        \[
            w_{1u} = w_1 \cdot \cos{ \beta_1 } = 
            272.03 \cdot 36.98^\circ = 
            217.32 \ м/с.
        \]

        \item Окружная составляющая относительной скорости на выходе из РК:
        \[
            w_{2u} = w_2 \cdot \cos{ \beta_2 } = 
            180.82 \cdot 63.66^\circ = 
            80.23 \ м/с.
        \]

        \item Абсолютная скорость на выходе из РК:
        \[
            c_2 = \frac{ c_{2a} }{ \sin{\alpha_2} } = 
            \frac{ 162.04 }{ \sin{ 38.0^\circ } } = 
            263.18\ м/с.
        \]

        \item Окружная составляющая скорости на выходе из РК:
        \[
            c_{2u} = \bar{c}_{2u} \cdot u_{2к} = 
            0.584 \cdot 355.29 = 
            207.382\ м/с.
        \]

        \item ГДФ температуры на входе в РК:
        \[
            \tau_1 = 1 - \frac{ k_{ср} - 1 }{ k_{ср} + 1 } \cdot \lambda_1^2 =  
            1 - \frac{ 1.386 - 1 }{ 1.386 + 1 } \cdot 0.574^2
        \]

        \item Статическая температура на входе в РК:
        \[
            T_1 = T_1^* \cdot \tau_1 = 288 \cdot 0.9467 = 
            272.648\ К.
        \]

        \item Скорость звука на входе в РК:
        \[
            a_1 = \sqrt{ k_{ср} \cdot R \cdot T_1 } = 
            \sqrt{ 1.386 \cdot 287.4 \cdot 272.65 } =
            329.54\ м/с.
        \]

        \item Число Маха в относительном движении на входе в РК:
        \[
            M_{w1\ ср} = \frac{ w_1 }{ a_1 } = \frac{ 272.03 }{ 329.54 } = 
            0.8255.
        \]

        \item Приведенная скорость на выходе из РК:
        \[
            \lambda_2 = \frac{ c_2 }{ a_{кр3} } =
            \frac{ 263.18 }{ 331.01 } = 
            0.7951.
        \]

        \item ГДФ температуры на выходе из РК:
        \[
            \tau_2 = 1 - \frac{ k_{ср} - 1 }{ k_{ср} + 1 } \cdot \lambda_2^2 =  
            1 - \frac{ 1.386 - 1 }{ 1.386 + 1 } \cdot 0.795^2
        \]

        \item Статическая температура на выходе из РК:
        \[
            T_2 = T_2^* \cdot \tau_2 = 328.17 \cdot 0.8978 =
            294.615\ К.
        \]

        \item Скорость звука на выходе из РК:
        \[
            a_2 = \sqrt{ k_{ср} \cdot R \cdot T_2 } = 
            \sqrt{ 1.386 \cdot 287.4 \cdot 294.62 } =
            342.56\ м/с.
        \]

         \item Число Маха в абсолютном движении на выходе из РК:
        \[
            M_{c2\ ср} = \frac{ c_2 }{ a_2 } = \frac{ 263.18 }{ 342.56 } = 
            0.7683.
        \]

    \end{enumerate}

    

    \subsection{Результаты.}

    
    
    \begin{longtable}{|p{0.6cm}|p{1.2cm}|p{1.2cm}|p{1.2cm}|p{1.2cm}|p{1.2cm}|p{1.2cm}|p{1.2cm}|p{1.3cm}|}
        \caption{Параметры ступеней коспрессора.}\\ \hline
        $N$ & $D_{1к}$, м & $u_{1к}$, м/с & $\bar{c}_{1a}$ & $\bar{H}_т$ & $\eta_{ад}^*$ &
        $c_{1a}$, м/с & $\Delta с_{1a\ ст}$, м/с & $\Delta с_{1a\ рк}$, м/с \\ \hline
%        
        1 & $0.631$ & $363.61$ &
        $0.45$ &
        $0.32$ & $0.85$ &
        $163.62$ & $3.16$ &
        $1.58$ \\ \hline
%        
        2 & $0.602$ & $346.98$ &
        $0.462$ &
        $0.334$ & $0.864$ &
        $160.46$ & $2.2$ &
        $1.1$ \\ \hline
%        
        3 & $0.582$ & $335.44$ &
        $0.472$ &
        $0.345$ & $0.875$ &
        $158.27$ & $1.57$ &
        $0.79$ \\ \hline
%        
        4 & $0.568$ & $326.95$ &
        $0.479$ &
        $0.355$ & $0.885$ &
        $156.7$ & $1.13$ &
        $0.57$ \\ \hline
%        
        5 & $0.556$ & $320.5$ &
        $0.485$ &
        $0.362$ & $0.892$ &
        $155.57$ & $0.81$ &
        $0.4$ \\ \hline
%        
        6 & $0.548$ & $315.48$ &
        $0.491$ &
        $0.366$ & $0.896$ &
        $154.75$ & $0.59$ &
        $0.29$ \\ \hline
%        
        7 & $0.541$ & $311.53$ &
        $0.495$ &
        $0.369$ & $0.899$ &
        $154.15$ & $0.43$ &
        $0.22$ \\ \hline
%        
        8 & $0.535$ & $308.37$ &
        $0.498$ &
        $0.369$ & $0.899$ &
        $153.71$ & $0.32$ &
        $0.16$ \\ \hline
%        
        9 & $0.531$ & $305.81$ &
        $0.502$ &
        $0.366$ & $0.896$ &
        $153.39$ & $0.23$ &
        $0.12$ \\ \hline
%        
        10 & $0.527$ & $303.73$ &
        $0.504$ &
        $0.361$ & $0.891$ &
        $153.16$ & $0.17$ &
        $0.09$ \\ \hline
%        
        11 & $0.524$ & $302.03$ &
        $0.506$ &
        $0.354$ & $0.884$ &
        $152.98$ & $0.13$ &
        $0.07$ \\ \hline
%        
        12 & $0.522$ & $300.64$ &
        $0.508$ &
        $0.343$ & $0.873$ &
        $152.85$ & $0.02$ &
        $0.01$ \\ \hline
%        
    \end{longtable}

    \begin{longtable}{|p{0.6cm}|p{1.3cm}|p{1.2cm}|p{1.2cm}|p{1.2cm}|p{1.2cm}|p{1.2cm}|p{1.2cm}|p{1.2cm}|}
        \caption{Параметры ступеней коспрессора.}\\ \hline
        $N$ & $\bar{c}_{2a}$ & $\bar{d}_{1вт}$ & $H_т$, КДж/кг & $L_z$, КДж/кг & $H_{ад}$, КДж/кг & $\Delta T^*$, К & $T_1^*$, К&
        $p_1^* \cdot 10^{-4}$, Па   \\ \hline
%        
        1 & $0.456$ & $0.5$ &
        $42.307$ &
        $41.461$ & $35.242$ &
        $40.17$ & $288$ &
        $9.934$  \\ \hline
%        
        2 & $0.467$ & $0.61$ &
        $40.171$ &
        $39.367$ & $34.0$ &
        $38.14$ & $328.17$ &
        $14.855$  \\ \hline
%        
        3 & $0.476$ & $0.685$ &
        $38.839$ &
        $38.062$ & $33.311$ &
        $36.88$ & $366.31$ &
        $20.944$  \\ \hline
%        
        4 & $0.482$ & $0.739$ &
        $37.897$ &
        $37.14$ & $32.851$ &
        $35.98$ & $403.19$ &
        $28.363$  \\ \hline
%        
        5 & $0.488$ & $0.78$ &
        $37.148$ &
        $36.405$ & $32.46$ &
        $35.27$ & $439.17$ &
        $37.262$  \\ \hline
%        
        6 & $0.493$ & $0.813$ &
        $36.473$ &
        $35.744$ & $32.043$ &
        $34.63$ & $474.44$ &
        $47.768$  \\ \hline
%        
        7 & $0.497$ & $0.838$ &
        $35.803$ &
        $35.087$ & $31.54$ &
        $33.99$ & $509.07$ &
        $59.978$  \\ \hline
%        
        8 & $0.5$ & $0.859$ &
        $35.082$ &
        $34.381$ & $30.906$ &
        $33.31$ & $543.06$ &
        $73.946$  \\ \hline
%        
        9 & $0.503$ & $0.876$ &
        $34.271$ &
        $33.586$ & $30.108$ &
        $32.54$ & $576.37$ &
        $89.665$  \\ \hline
%        
        10 & $0.505$ & $0.89$ &
        $33.34$ &
        $32.674$ & $29.125$ &
        $31.66$ & $608.91$ &
        $107.06$  \\ \hline
%        
        11 & $0.507$ & $0.901$ &
        $32.265$ &
        $31.62$ & $27.942$ &
        $30.63$ & $640.57$ &
        $125.974$  \\ \hline
%        
        12 & $0.509$ & $0.91$ &
        $31.025$ &
        $30.404$ & $26.55$ &
        $29.46$ & $671.2$ &
        $146.165$  \\ \hline
%        
    \end{longtable}

     \begin{longtable}{|p{0.6cm}|p{1.2cm}|p{1.2cm}|p{1.2cm}|p{1.2cm}|p{1.2cm}|p{1.2cm}|p{1.2cm}|p{1.2cm}|}
        \caption{Параметры ступеней коспрессора.}\\ \hline
        $N$ & $a_{кр1}$, м/с & $\bar{r}_{ср1}$ & $\bar{c}_{u1}$ & $\alpha_1, ^\circ$ & $\alpha_2,\ ^\circ$ &
        $\varepsilon_{рк},\ ^\circ$ & $\varepsilon_{на},\ ^\circ$ & $w_1$, м/с  \\ \hline
%        
        1 & $310.09$ & $0.791$ &
        $0.193$ & $66.8$ & $38.0$ &
        $26.68$ & $27.28$ &
        $272.03$ \\ \hline
%        
        2 & $331.01$ & $0.828$ &
        $0.213$ & $65.29$ & $37.66$ &
        $26.13$ & $26.64$ &
        $267.16$ \\ \hline
%        
        3 & $349.72$ & $0.857$ &
        $0.227$ & $64.3$ & $37.41$ &
        $25.78$ & $26.19$ &
        $263.99$ \\ \hline
%        
        4 & $366.9$ & $0.879$ &
        $0.238$ & $63.6$ & $37.23$ &
        $25.53$ & $25.82$ &
        $261.74$ \\ \hline
%        
        5 & $382.92$ & $0.897$ &
        $0.247$ & $63.05$ & $37.11$ &
        $25.29$ & $25.47$ &
        $260.02$ \\ \hline
%        
        6 & $398.0$ & $0.911$ &
        $0.254$ & $62.58$ & $37.05$ &
        $25.02$ & $25.09$ &
        $258.58$ \\ \hline
%        
        7 & $412.27$ & $0.923$ &
        $0.261$ & $62.15$ & $37.05$ &
        $24.7$ & $24.67$ &
        $257.29$ \\ \hline
%        
        8 & $425.82$ & $0.932$ &
        $0.268$ & $61.72$ & $37.1$ &
        $24.3$ & $24.16$ &
        $256.03$ \\ \hline
%        
        9 & $438.68$ & $0.94$ &
        $0.275$ & $61.26$ & $37.2$ &
        $23.81$ & $23.56$ &
        $254.71$ \\ \hline
%        
        10 & $450.89$ & $0.946$ &
        $0.282$ & $60.76$ & $37.35$ &
        $23.21$ & $22.85$ &
        $253.28$ \\ \hline
%        
        11 & $462.47$ & $0.952$ &
        $0.29$ & $60.2$ & $37.55$ &
        $22.49$ & $22.02$ &
        $251.68$ \\ \hline
%        
        12 & $473.4$ & $0.956$ &
        $0.299$ & $59.57$ & $37.81$ &
        $21.65$ & $21.19$ &
        $249.89$ \\ \hline
%        
    \end{longtable}

    \begin{longtable}{|p{0.6cm}|p{1.1cm}|p{1.1cm}|p{1.1cm}|p{1.1cm}|p{1.1cm}| p{1.1cm}|p{1.1cm}|p{1.1cm}|}
        \caption{Параметры ступеней коспрессора.}\\ \hline
        $N$ & $w_2$, м/с & $c_1$, м/с & $c_2$, м/с & $\tau_1$ & $T_1$, К & $a_1$, м/с & $M_{w1}$ & $\lambda_{c2}$ \\ \hline
%        
        1 & $180.82$ &
        $178.02$ & $263.18$ & $0.947$ &
        $272.65$ & $329.54$ & $0.825$ &
        $0.795$\\ \hline
%        
        2 & $178.79$ &
        $176.64$ & $260.84$ & $0.954$ &
        $313.05$ & $353.12$ & $0.757$ &
        $0.746$\\ \hline
%        
        3 & $177.35$ &
        $175.65$ & $259.25$ & $0.959$ &
        $351.36$ & $374.1$ & $0.706$ &
        $0.707$\\ \hline
%        
        4 & $176.34$ &
        $174.96$ & $258.1$ & $0.963$ &
        $388.36$ & $393.3$ & $0.665$ &
        $0.674$\\ \hline
%        
        5 & $175.68$ &
        $174.53$ & $257.17$ & $0.966$ &
        $424.41$ & $411.15$ & $0.632$ &
        $0.646$\\ \hline
%        
        6 & $175.29$ &
        $174.33$ & $256.33$ & $0.969$ &
        $459.72$ & $427.91$ & $0.604$ &
        $0.622$\\ \hline
%        
        7 & $175.15$ &
        $174.35$ & $255.49$ & $0.971$ &
        $494.34$ & $443.73$ & $0.58$ &
        $0.6$\\ \hline
%        
        8 & $175.23$ &
        $174.55$ & $254.58$ & $0.973$ &
        $528.3$ & $458.72$ & $0.558$ &
        $0.58$\\ \hline
%        
        9 & $175.52$ &
        $174.95$ & $253.54$ & $0.974$ &
        $561.54$ & $472.93$ & $0.539$ &
        $0.562$\\ \hline
%        
        10 & $176.01$ &
        $175.53$ & $252.33$ & $0.975$ &
        $593.99$ & $486.4$ & $0.521$ &
        $0.546$\\ \hline
%        
        11 & $176.7$ &
        $176.29$ & $250.91$ & $0.976$ &
        $625.51$ & $499.14$ & $0.504$ &
        $0.53$\\ \hline
%        
        12 & $177.64$ &
        $177.26$ & $249.29$ & $0.977$ &
        $655.98$ & $511.16$ & $0.489$ &
        $0.515$\\ \hline
%        
    \end{longtable}

    



\end{document}